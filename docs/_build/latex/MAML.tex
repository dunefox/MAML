% Generated by Sphinx.
\def\sphinxdocclass{report}
\newif\ifsphinxKeepOldNames \sphinxKeepOldNamestrue
\documentclass[letterpaper,10pt,english]{sphinxmanual}
\usepackage{iftex}

\ifPDFTeX
  \usepackage[utf8]{inputenc}
\fi
\ifdefined\DeclareUnicodeCharacter
  \DeclareUnicodeCharacter{00A0}{\nobreakspace}
\fi
\usepackage{cmap}
\usepackage[T1]{fontenc}
\usepackage{amsmath,amssymb,amstext}
\usepackage{babel}
\usepackage{times}
\usepackage[Bjarne]{fncychap}
\usepackage{longtable}
\usepackage{sphinx}
\usepackage{multirow}
\usepackage{eqparbox}


\addto\captionsenglish{\renewcommand{\figurename}{Fig.\@ }}
\addto\captionsenglish{\renewcommand{\tablename}{Table }}
\SetupFloatingEnvironment{literal-block}{name=Listing }

\addto\extrasenglish{\def\pageautorefname{page}}

\setcounter{tocdepth}{2}


\title{MAML Documentation}
\date{Oct 13, 2016}
\release{0.1.1}
\author{Dirk - André Deckert}
\newcommand{\sphinxlogo}{}
\renewcommand{\releasename}{Release}
\makeindex

\makeatletter
\def\PYG@reset{\let\PYG@it=\relax \let\PYG@bf=\relax%
    \let\PYG@ul=\relax \let\PYG@tc=\relax%
    \let\PYG@bc=\relax \let\PYG@ff=\relax}
\def\PYG@tok#1{\csname PYG@tok@#1\endcsname}
\def\PYG@toks#1+{\ifx\relax#1\empty\else%
    \PYG@tok{#1}\expandafter\PYG@toks\fi}
\def\PYG@do#1{\PYG@bc{\PYG@tc{\PYG@ul{%
    \PYG@it{\PYG@bf{\PYG@ff{#1}}}}}}}
\def\PYG#1#2{\PYG@reset\PYG@toks#1+\relax+\PYG@do{#2}}

\expandafter\def\csname PYG@tok@gd\endcsname{\def\PYG@tc##1{\textcolor[rgb]{0.63,0.00,0.00}{##1}}}
\expandafter\def\csname PYG@tok@gu\endcsname{\let\PYG@bf=\textbf\def\PYG@tc##1{\textcolor[rgb]{0.50,0.00,0.50}{##1}}}
\expandafter\def\csname PYG@tok@gt\endcsname{\def\PYG@tc##1{\textcolor[rgb]{0.00,0.27,0.87}{##1}}}
\expandafter\def\csname PYG@tok@gs\endcsname{\let\PYG@bf=\textbf}
\expandafter\def\csname PYG@tok@gr\endcsname{\def\PYG@tc##1{\textcolor[rgb]{1.00,0.00,0.00}{##1}}}
\expandafter\def\csname PYG@tok@cm\endcsname{\let\PYG@it=\textit\def\PYG@tc##1{\textcolor[rgb]{0.25,0.50,0.56}{##1}}}
\expandafter\def\csname PYG@tok@vg\endcsname{\def\PYG@tc##1{\textcolor[rgb]{0.73,0.38,0.84}{##1}}}
\expandafter\def\csname PYG@tok@vi\endcsname{\def\PYG@tc##1{\textcolor[rgb]{0.73,0.38,0.84}{##1}}}
\expandafter\def\csname PYG@tok@mh\endcsname{\def\PYG@tc##1{\textcolor[rgb]{0.13,0.50,0.31}{##1}}}
\expandafter\def\csname PYG@tok@cs\endcsname{\def\PYG@tc##1{\textcolor[rgb]{0.25,0.50,0.56}{##1}}\def\PYG@bc##1{\setlength{\fboxsep}{0pt}\colorbox[rgb]{1.00,0.94,0.94}{\strut ##1}}}
\expandafter\def\csname PYG@tok@ge\endcsname{\let\PYG@it=\textit}
\expandafter\def\csname PYG@tok@vc\endcsname{\def\PYG@tc##1{\textcolor[rgb]{0.73,0.38,0.84}{##1}}}
\expandafter\def\csname PYG@tok@il\endcsname{\def\PYG@tc##1{\textcolor[rgb]{0.13,0.50,0.31}{##1}}}
\expandafter\def\csname PYG@tok@go\endcsname{\def\PYG@tc##1{\textcolor[rgb]{0.20,0.20,0.20}{##1}}}
\expandafter\def\csname PYG@tok@cp\endcsname{\def\PYG@tc##1{\textcolor[rgb]{0.00,0.44,0.13}{##1}}}
\expandafter\def\csname PYG@tok@gi\endcsname{\def\PYG@tc##1{\textcolor[rgb]{0.00,0.63,0.00}{##1}}}
\expandafter\def\csname PYG@tok@gh\endcsname{\let\PYG@bf=\textbf\def\PYG@tc##1{\textcolor[rgb]{0.00,0.00,0.50}{##1}}}
\expandafter\def\csname PYG@tok@ni\endcsname{\let\PYG@bf=\textbf\def\PYG@tc##1{\textcolor[rgb]{0.84,0.33,0.22}{##1}}}
\expandafter\def\csname PYG@tok@nl\endcsname{\let\PYG@bf=\textbf\def\PYG@tc##1{\textcolor[rgb]{0.00,0.13,0.44}{##1}}}
\expandafter\def\csname PYG@tok@nn\endcsname{\let\PYG@bf=\textbf\def\PYG@tc##1{\textcolor[rgb]{0.05,0.52,0.71}{##1}}}
\expandafter\def\csname PYG@tok@no\endcsname{\def\PYG@tc##1{\textcolor[rgb]{0.38,0.68,0.84}{##1}}}
\expandafter\def\csname PYG@tok@na\endcsname{\def\PYG@tc##1{\textcolor[rgb]{0.25,0.44,0.63}{##1}}}
\expandafter\def\csname PYG@tok@nb\endcsname{\def\PYG@tc##1{\textcolor[rgb]{0.00,0.44,0.13}{##1}}}
\expandafter\def\csname PYG@tok@nc\endcsname{\let\PYG@bf=\textbf\def\PYG@tc##1{\textcolor[rgb]{0.05,0.52,0.71}{##1}}}
\expandafter\def\csname PYG@tok@nd\endcsname{\let\PYG@bf=\textbf\def\PYG@tc##1{\textcolor[rgb]{0.33,0.33,0.33}{##1}}}
\expandafter\def\csname PYG@tok@ne\endcsname{\def\PYG@tc##1{\textcolor[rgb]{0.00,0.44,0.13}{##1}}}
\expandafter\def\csname PYG@tok@nf\endcsname{\def\PYG@tc##1{\textcolor[rgb]{0.02,0.16,0.49}{##1}}}
\expandafter\def\csname PYG@tok@si\endcsname{\let\PYG@it=\textit\def\PYG@tc##1{\textcolor[rgb]{0.44,0.63,0.82}{##1}}}
\expandafter\def\csname PYG@tok@s2\endcsname{\def\PYG@tc##1{\textcolor[rgb]{0.25,0.44,0.63}{##1}}}
\expandafter\def\csname PYG@tok@nt\endcsname{\let\PYG@bf=\textbf\def\PYG@tc##1{\textcolor[rgb]{0.02,0.16,0.45}{##1}}}
\expandafter\def\csname PYG@tok@nv\endcsname{\def\PYG@tc##1{\textcolor[rgb]{0.73,0.38,0.84}{##1}}}
\expandafter\def\csname PYG@tok@s1\endcsname{\def\PYG@tc##1{\textcolor[rgb]{0.25,0.44,0.63}{##1}}}
\expandafter\def\csname PYG@tok@ch\endcsname{\let\PYG@it=\textit\def\PYG@tc##1{\textcolor[rgb]{0.25,0.50,0.56}{##1}}}
\expandafter\def\csname PYG@tok@m\endcsname{\def\PYG@tc##1{\textcolor[rgb]{0.13,0.50,0.31}{##1}}}
\expandafter\def\csname PYG@tok@gp\endcsname{\let\PYG@bf=\textbf\def\PYG@tc##1{\textcolor[rgb]{0.78,0.36,0.04}{##1}}}
\expandafter\def\csname PYG@tok@sh\endcsname{\def\PYG@tc##1{\textcolor[rgb]{0.25,0.44,0.63}{##1}}}
\expandafter\def\csname PYG@tok@ow\endcsname{\let\PYG@bf=\textbf\def\PYG@tc##1{\textcolor[rgb]{0.00,0.44,0.13}{##1}}}
\expandafter\def\csname PYG@tok@sx\endcsname{\def\PYG@tc##1{\textcolor[rgb]{0.78,0.36,0.04}{##1}}}
\expandafter\def\csname PYG@tok@bp\endcsname{\def\PYG@tc##1{\textcolor[rgb]{0.00,0.44,0.13}{##1}}}
\expandafter\def\csname PYG@tok@c1\endcsname{\let\PYG@it=\textit\def\PYG@tc##1{\textcolor[rgb]{0.25,0.50,0.56}{##1}}}
\expandafter\def\csname PYG@tok@o\endcsname{\def\PYG@tc##1{\textcolor[rgb]{0.40,0.40,0.40}{##1}}}
\expandafter\def\csname PYG@tok@kc\endcsname{\let\PYG@bf=\textbf\def\PYG@tc##1{\textcolor[rgb]{0.00,0.44,0.13}{##1}}}
\expandafter\def\csname PYG@tok@c\endcsname{\let\PYG@it=\textit\def\PYG@tc##1{\textcolor[rgb]{0.25,0.50,0.56}{##1}}}
\expandafter\def\csname PYG@tok@mf\endcsname{\def\PYG@tc##1{\textcolor[rgb]{0.13,0.50,0.31}{##1}}}
\expandafter\def\csname PYG@tok@err\endcsname{\def\PYG@bc##1{\setlength{\fboxsep}{0pt}\fcolorbox[rgb]{1.00,0.00,0.00}{1,1,1}{\strut ##1}}}
\expandafter\def\csname PYG@tok@mb\endcsname{\def\PYG@tc##1{\textcolor[rgb]{0.13,0.50,0.31}{##1}}}
\expandafter\def\csname PYG@tok@ss\endcsname{\def\PYG@tc##1{\textcolor[rgb]{0.32,0.47,0.09}{##1}}}
\expandafter\def\csname PYG@tok@sr\endcsname{\def\PYG@tc##1{\textcolor[rgb]{0.14,0.33,0.53}{##1}}}
\expandafter\def\csname PYG@tok@mo\endcsname{\def\PYG@tc##1{\textcolor[rgb]{0.13,0.50,0.31}{##1}}}
\expandafter\def\csname PYG@tok@kd\endcsname{\let\PYG@bf=\textbf\def\PYG@tc##1{\textcolor[rgb]{0.00,0.44,0.13}{##1}}}
\expandafter\def\csname PYG@tok@mi\endcsname{\def\PYG@tc##1{\textcolor[rgb]{0.13,0.50,0.31}{##1}}}
\expandafter\def\csname PYG@tok@kn\endcsname{\let\PYG@bf=\textbf\def\PYG@tc##1{\textcolor[rgb]{0.00,0.44,0.13}{##1}}}
\expandafter\def\csname PYG@tok@cpf\endcsname{\let\PYG@it=\textit\def\PYG@tc##1{\textcolor[rgb]{0.25,0.50,0.56}{##1}}}
\expandafter\def\csname PYG@tok@kr\endcsname{\let\PYG@bf=\textbf\def\PYG@tc##1{\textcolor[rgb]{0.00,0.44,0.13}{##1}}}
\expandafter\def\csname PYG@tok@s\endcsname{\def\PYG@tc##1{\textcolor[rgb]{0.25,0.44,0.63}{##1}}}
\expandafter\def\csname PYG@tok@kp\endcsname{\def\PYG@tc##1{\textcolor[rgb]{0.00,0.44,0.13}{##1}}}
\expandafter\def\csname PYG@tok@w\endcsname{\def\PYG@tc##1{\textcolor[rgb]{0.73,0.73,0.73}{##1}}}
\expandafter\def\csname PYG@tok@kt\endcsname{\def\PYG@tc##1{\textcolor[rgb]{0.56,0.13,0.00}{##1}}}
\expandafter\def\csname PYG@tok@sc\endcsname{\def\PYG@tc##1{\textcolor[rgb]{0.25,0.44,0.63}{##1}}}
\expandafter\def\csname PYG@tok@sb\endcsname{\def\PYG@tc##1{\textcolor[rgb]{0.25,0.44,0.63}{##1}}}
\expandafter\def\csname PYG@tok@k\endcsname{\let\PYG@bf=\textbf\def\PYG@tc##1{\textcolor[rgb]{0.00,0.44,0.13}{##1}}}
\expandafter\def\csname PYG@tok@se\endcsname{\let\PYG@bf=\textbf\def\PYG@tc##1{\textcolor[rgb]{0.25,0.44,0.63}{##1}}}
\expandafter\def\csname PYG@tok@sd\endcsname{\let\PYG@it=\textit\def\PYG@tc##1{\textcolor[rgb]{0.25,0.44,0.63}{##1}}}

\def\PYGZbs{\char`\\}
\def\PYGZus{\char`\_}
\def\PYGZob{\char`\{}
\def\PYGZcb{\char`\}}
\def\PYGZca{\char`\^}
\def\PYGZam{\char`\&}
\def\PYGZlt{\char`\<}
\def\PYGZgt{\char`\>}
\def\PYGZsh{\char`\#}
\def\PYGZpc{\char`\%}
\def\PYGZdl{\char`\$}
\def\PYGZhy{\char`\-}
\def\PYGZsq{\char`\'}
\def\PYGZdq{\char`\"}
\def\PYGZti{\char`\~}
% for compatibility with earlier versions
\def\PYGZat{@}
\def\PYGZlb{[}
\def\PYGZrb{]}
\makeatother

\renewcommand\PYGZsq{\textquotesingle}

\begin{document}

\maketitle
\tableofcontents
\phantomsection\label{index::doc}


Contents:


\chapter{Mathematics and Applications of Machine Learning}
\label{readme:mathematics-and-applications-of-machine-learning}\label{readme::doc}\begin{itemize}
\item {} 
\textbf{Institution:} Mathematical Institute, LMU Munich

\item {} 
\textbf{Term:} Winter semester 2016/17

\item {} 
\textbf{Lecturer:} Dirk - André Deckert, \href{mailto:deckert@math.lmu.de}{deckert@math.lmu.de}

\item {} 
\textbf{Time:} Wednesday, 14:00-16:00

\item {} 
\textbf{Location:} Lecture Hall A 027

\end{itemize}


\section{Description}
\label{readme:description}
This course will give an introduction to selected topics on machine learning.
We will start from the basic perceptron and proceed with support vector
machines, multi-layer networks, and aspects of deep learning. The mathematical
discussion will focus on machine learning as an optimization problem. As
regards applications, it is the goal of this lecture and its tutorials to
implement several applications of the discussed algorithms in Python.
Therefore, basic knowledge in Python programming and access to a computer with
a Python development environment is expected -- and will be required to
complete the exercises. If time permits and depending on the interest, we may
furthermore discuss aspects of recurrent networks and reinforcement learning.


\section{Disclaimer}
\label{readme:disclaimer}
As always, these notes have been written in quite a haste during the semester
and will contain lots of typos. If you find a typo please help to improve these
notes and report it with a precise reference (URL, equation number, etc.) to my
email address above. To my students: This material is in most parts less
detailed than the discussion in the lecture. Please let me know if and where,
you feel the notes came too short. Thanks in advance!


\chapter{Text-books and supplementary material}
\label{supplementary::doc}\label{supplementary:text-books-and-supplementary-material}
The lecture will not follow one particular text-book. Rather we will pick out
some topics from here and there. However, the following references provide some
sources:
\begin{itemize}
\item {} 
For a brief introduction into python see \phantomsection\label{supplementary:id1}{\hyperref[supplementary:gael\string-scipy\string-2016]{\sphinxcrossref{{[}GGV16{]}}}}.

\item {} 
Two standard references on \emph{machine learning} are
\phantomsection\label{supplementary:id2}{\hyperref[supplementary:bishop\string-pattern\string-2007]{\sphinxcrossref{{[}Bis07{]}}}} and \phantomsection\label{supplementary:id3}{\hyperref[supplementary:russell\string-artificial\string-2010]{\sphinxcrossref{{[}RN10{]}}}}.

\item {} 
A really nice hands-on approach on deep learning is
\phantomsection\label{supplementary:id4}{\hyperref[supplementary:nielsen\string-neural\string-2015]{\sphinxcrossref{{[}Nie15{]}}}}.  We will follow those notes a great deal.

\item {} 
Another good reference on \emph{machine learning} with a more mathematical focus is
\phantomsection\label{supplementary:id5}{\hyperref[supplementary:mohri\string-foundations\string-2012]{\sphinxcrossref{{[}Moh12{]}}}}.

\end{itemize}




\chapter{Steinbruch}
\label{test/Test_reST::doc}\label{test/Test_reST:steinbruch}
Ok, here is a formula with a label
\phantomsection\label{test/Test_reST:equation-test}\begin{equation}\label{test/Test_reST-test}
\begin{split}f(x) = W x + b\end{split}
\end{equation}\begin{equation*}
\begin{split}\begin{align}
    F(b) - F(a) &= \int_a^b f(x) dx
    \label{test1}
\end{align}\end{split}
\end{equation*}
And we can refer to this equation as Eq. \eqref{test/Test_reST-test}. And this is an inline formula \(f(x)\). It is also possible to refer to \(\eqref{test1}\).

We can immediately plot the function. Here, for \(W=2\) and \(b=1\):

\begin{Verbatim}[commandchars=\\\{\}]
\PYG{k+kn}{import} \PYG{n+nn}{matplotlib.pyplot} \PYG{k+kn}{as} \PYG{n+nn}{plt}
\PYG{k+kn}{import} \PYG{n+nn}{numpy} \PYG{k+kn}{as} \PYG{n+nn}{np}
\PYG{n}{W} \PYG{o}{=} \PYG{l+m+mi}{2}
\PYG{n}{b} \PYG{o}{=} \PYG{l+m+mi}{1}
\PYG{n}{x} \PYG{o}{=} \PYG{n}{np}\PYG{o}{.}\PYG{n}{linspace}\PYG{p}{(}\PYG{l+m+mi}{0}\PYG{p}{,} \PYG{l+m+mi}{10}\PYG{p}{)}
\PYG{n}{plt}\PYG{o}{.}\PYG{n}{plot}\PYG{p}{(}\PYG{n}{x}\PYG{p}{,} \PYG{n}{W}\PYG{o}{*}\PYG{n}{x}\PYG{o}{+}\PYG{n}{b}\PYG{p}{)}
\PYG{n}{plt}\PYG{o}{.}\PYG{n}{show}\PYG{p}{(}\PYG{p}{)}
\end{Verbatim}

\noindent\sphinxincludegraphics{{matplot}.pdf}

For the plot we have used the code:

\begin{Verbatim}[commandchars=\\\{\},numbers=left,firstnumber=1,stepnumber=1]
\PYG{k}{def} \PYG{n+nf}{sum}\PYG{p}{(}\PYG{n}{a}\PYG{p}{,} \PYG{n}{b}\PYG{p}{)}\PYG{p}{:}
    \PYG{k}{return} \PYG{n}{a} \PYG{o}{+} \PYG{n}{b}
\end{Verbatim}

Or files can be sourced:

\begin{Verbatim}[commandchars=\\\{\},numbers=left,firstnumber=1,stepnumber=1]
\PYG{k+kn}{import} \PYG{n+nn}{matplotlib}\PYG{n+nn}{.}\PYG{n+nn}{pyplot} \PYG{k}{as} \PYG{n+nn}{plt}
\PYG{k+kn}{import} \PYG{n+nn}{numpy} \PYG{k}{as} \PYG{n+nn}{np}
\PYG{n}{W} \PYG{o}{=} \PYG{l+m+mi}{2}
\PYG{n}{b} \PYG{o}{=} \PYG{l+m+mi}{1}
\PYG{n}{x} \PYG{o}{=} \PYG{n}{np}\PYG{o}{.}\PYG{n}{linspace}\PYG{p}{(}\PYG{l+m+mi}{0}\PYG{p}{,} \PYG{l+m+mi}{10}\PYG{p}{)}
\PYG{n}{plt}\PYG{o}{.}\PYG{n}{plot}\PYG{p}{(}\PYG{n}{x}\PYG{p}{,} \PYG{n}{W}\PYG{o}{*}\PYG{n}{x}\PYG{o}{+}\PYG{n}{b}\PYG{p}{)}
\PYG{n}{plt}\PYG{o}{.}\PYG{n}{show}\PYG{p}{(}\PYG{p}{)}
\end{Verbatim}

This is something taken from \phantomsection\label{test/Test_reST:id1}{\hyperref[test/Test_reST:testcit2016]{\sphinxcrossref{{[}TESTCIT2016{]}}}}

Here is some JavaScript

And here an interactive plot:


\section{This is a section}
\label{test/Test_reST:sec-thisone}\label{test/Test_reST:this-is-a-section}
How can we get numbers? Well, like this...
This is {\hyperref[test/Test_reST:sec\string-thisone]{\sphinxcrossref{\DUrole{std,std-ref,std,std-ref}{3.1}}}}!

Here is a proof that we want to be collapsed:

\textbf{Theorem.}

\textbf{Proof} First we compute
\phantomsection\label{test/Test_reST:equation-test2}\begin{equation}\label{test/Test_reST-test2}
\begin{split}f(x) = W x + b\end{split}
\end{equation}
Toggle Proof


\chapter{References}
\label{test/Test_reST:references}

\chapter{Indices and tables}
\label{index:indices-and-tables}\begin{itemize}
\item {} 
\DUrole{xref,std,std-ref}{genindex}

\item {} 
\DUrole{xref,std,std-ref}{modindex}

\item {} 
\DUrole{xref,std,std-ref}{search}

\end{itemize}

\begin{thebibliography}{TESTCIT2016}
\bibitem[Bis07]{Bis07}{\phantomsection\label{supplementary:bishop-pattern-2007} 
Christopher Bishop. \emph{Pattern Recognition and Machine Learning}. Springer, New York, 1st ed. 2006. corr. 2nd printing 2011 edition, 2007. ISBN 978-0-387-31073-2. URL: \url{https://www.amazon.de/Pattern-Recognition-Learning-Information-Statistics/dp/0387310738/ref=sr\_1\_1?ie=UTF8\&qid=1475680474\&sr=8-1\&keywords=Pattern+Recognition+and+Machine+Learning}.
}
\bibitem[GGV16]{GGV16}{\phantomsection\label{supplementary:gael-scipy-2016} 
Varoquaux Gaël, Emmanuelle Gouillart, and Olav Vahtras. Scipy Lecture Notes — Scipy lecture notes. 2016. URL: \url{http://www.scipy-lectures.org/index.html}.
}
\bibitem[Moh12]{Moh12}{\phantomsection\label{supplementary:mohri-foundations-2012} 
Mohri. \emph{Foundations of Machine Learning}. Mit University Press Group Ltd, Cambridge, MA, September 2012. ISBN 978-0-262-01825-8. URL: \url{https://www.amazon.de/Foundations-Machine-Learning-Adaptive-Computation/dp/026201825X/ref=sr\_1\_fkmr0\_1?ie=UTF8\&qid=1475680504\&sr=8-1-fkmr0\&keywords=Foundations+of+Machine+Learning++++33+..+}.
}
\bibitem[Nie15]{Nie15}{\phantomsection\label{supplementary:nielsen-neural-2015} 
Michael A. Nielsen. Neural Networks and Deep Learning. 2015. URL: \url{http://neuralnetworksanddeeplearning.com}.
}
\bibitem[RN10]{RN10}{\phantomsection\label{supplementary:russell-artificial-2010} 
Stuart J. Russell and Peter Norvig. \emph{Artificial Intelligence: A Modern Approach}. Prentice Hall, Upper Saddle River, 00003 edition, February 2010. ISBN 978-0-13-604259-4. URL: \url{https://www.amazon.de/Artificial-Intelligence-Modern-Approach-Prentice/dp/0136042597/ref=sr\_1\_1?ie=UTF8\&qid=1475680391\&sr=8-1\&keywords=russell}.
}
\bibitem[TESTCIT2016]{TESTCIT2016}{\phantomsection\label{test/Test_reST:testcit2016} 
This is a great paper.
}
\end{thebibliography}



\renewcommand{\indexname}{Index}
\printindex
\end{document}
